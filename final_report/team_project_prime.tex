\documentclass{article}
\title{Final Report}
\author{By Project Prime}
\begin{document}
 \maketitle
 \section{\underline{Introduction}}
In this project, we were tasked with building a multi-host file synchroniser consisting of three components: a server, and mobile and desktop client. The server is the "hub" whilst the clients are "spokes". The server was created using Node.js, where it acted as an interface between both clients and the database (which was created using MongoDB). In addition, this server processed HTTP GET and POST requests to deliver/updates files on the client's behalf. The mobile client was made using Java and XML on Android Studio, whereas the desktop client was created using Javascript, HTML and CSS on Electron. The result of this project is a fully-functioning file synchroniser that interacts with and manages multiple, heterogeneous clients via a Node.js server. Furthermore, this synchroniser possesses the necessary algorithms (e.g. rsync) to resolve file conflicts caused by clients. 

\section{\underline{Review}}

\section{\underline{Requirements and Design}}

\subsection{\underline{Server Application}}

\subsection{\underline{Desktop and Mobile Client}}

\section{\underline{Implementation}}

\subsection{\underline{Server Application}}

\subsection{\underline{Desktop Client}}
Yusaf and Sandipan built the desktop application using the Electron framework; thus, its implementation primarily involved HTML, Javascript and CSS coding.

\subsection{\underline{Mobile Client}}

\section{\underline{Teamwork}}
Project Prime consists of six members: Yusaf, Sandipan, Saloni, Shefali, Cameron and Manny. To balance the workload, the team was divided into three subgroups of two teammates, and each subgroup was assigned to the development of one component. As a result, Yusaf and Sandipan were assigned to the desktop client, Saloni and Shefali were assigned to the server application, and Manny and Cameron were assigned to the mobile client. Despite this style of work delegation, members from other subgroups were still able to intervene in the development of a component that they weren't assigned to, in order to fix issues that the assigned subgroup couldn't correct, for example. Moving onto communication, the team remotely communicated using an instant messaging app known as "Whatsapp", which allowed us to arrange group sessions at a certain date-and-time, notify teammates of recent pull requests, etc. Furthermore, the sessions were conducted in booked group study rooms at least once per week, and these sessions involved group discussion and coding of all components. 

\section{\underline{Evaluation}}
From start to finish, the team experienced  positive and negative events of the project. To start with the positives, the biggest highlight of this project was the strong team communication, as each member confidently conveyed their own thoughts and opinions during group discussions in physical meetings and WhatsApp. In addition, when disagreements regarding the project arose, debates were conducted in a respectful manner by not saying rude or offensive statements. Each member also made the effort to attend each group meeting throughout the project, unless there was a genuine reason for their absence. Another positive was our ability to quickly adapt to changing circumstances. For instance, as initially planned, we were going to contain files in an SQL database, but upon discovering that SQL databases were not ideal for file storage, we got stuck in a dillemma. Fortunately, after thorough research, we decided to use a Mongo database since the latter is a document-oriented database system that's recommended for storing files. However, moving onto the negatives, the biggest drawback was how weak the plan was since each member was inexperienced in developing a multi-host file synchroniser, which led to us being unsure of how to approach the task. In addition, the level of inexperience formerly impacted our confidence to build the system. However, as the project progressed, the Internet became a major learning tool to gain new skills to become confident in achieving the project aims.

Next,



In conclusion, this project caused the realisation that despite our computer science backgrounds, our technical prowess is still very basic and there is much for us to learn. In response to this discovery, we will commit to thoroughly relearning and practising how to use different technologies in our spare time, including programming languages and Git (e.g. BitBucket). In retrospect, if this project was repeated, our different approach would be to partner stronger members with novice ones, in terms of technological skill; thus, providing novice members with the opportunity to enhance their skills by learning whilst working on the job.   

\section{\underline{Peer Assessment}}
\begin{tabular}{|p{2cm}|p{2cm}|p{2cm}|p{2cm}|p{2cm}|p{2cm}|}
\hline
\multicolumn{6}{|c|}{\textbf{Peer Assessment of Project Prime}} \\
\hline
\textbf{Yusaf} & \textbf{Sandipan} & \textbf{Saloni} & \textbf{Shefali} & \textbf{Cameron} & \textbf{Manny} \\
\hline
0 & 0 & 0 & 0 & 0 & 0 \\
\hline
\end{tabular}
	
\section{\underline{References}}
\end{document}