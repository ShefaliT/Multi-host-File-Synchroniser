\documentclass{article}
\usepackage{parskip}
\title{Final Report}
\author{By Project Prime}
\begin{document}
 \maketitle
 \section{\underline{Introduction}}
In this project, we were tasked with building a multi-host file synchroniser consisting of three components: a server, and mobile and desktop client. The server is the "hub" whilst the clients are "spokes". The server was created using Node.js, where it acted as an interface between both clients and the database (which was created using MongoDB). In addition, this server processed HTTP GET and POST requests to deliver/updates files on the client's behalf. The mobile client was made using Java and XML on Android Studio, whereas the desktop client was created using Javascript, HTML and CSS on Electron. The result of this project is a fully-functioning file synchroniser that interacts with and manages multiple, heterogeneous clients via a Node.js server. Furthermore, this synchroniser possesses the necessary algorithms (e.g. rsync) to resolve file conflicts caused by clients. 

\section{\underline{Review}}

\section{\underline{Requirements and Design}}
Requirements have been derived for each component. The requirements have been analysed using the MoSCoW prioritisation technique (Madsen, 2017); as a result, each requirement has been placed into one of the following categories: \textit{Must Have} (critical requirements with the highest priority), \textit{Should Have} (Important but unnecessary requirements for the final product), \textit{Could Have} (Lowest-priority requirements that would be implemented if time permits) and \textit{Won't Have} (Least critical requirements that are unrequired for project success). In addition, the subsections will show the designs that reflect the corresponding component.

\subsection{\underline{Desktop Client}}
\begin{tabular}{|p{3cm}|p{5cm}|p{4cm}|}
\hline
\multicolumn{3}{|c|}{\textbf{MoSCoW requirements for The Desktop Client}} \\
\hline
\textbf{Requirement No.} & \textbf{Requirement} & \textbf{Priority}\\
\hline
1 & Must be able to communicate with the Server Application & Must Have \\
\hline
2 & Has to be able to upload files & Must Have \\
\hline
3 & Has to be able to download files from database via server & Must Have \\
\hline
4 & Make use of a file management system to upload any file from the user's system & Must Have \\
\hline
5 & Check for updates regularly so that the desktop client is in sync with the Server and Mobile Client & Must Have\\
\hline
6 & Display the current list of files from the database in the centre of the page & Should Have \\
\hline
7 & Include a delete file functionality   & Should Have \\
\hline
8 & Provide a way to track and check recent changes to files & Could Have\\
\hline
9 & Include a search bar to quickly find files if the list of files is too large & Could Have \\
\hline
10 & Have User profiles for personalised access & Won't Have \\
\hline
\end{tabular}

\subsection{\underline{Mobile Client}}
\begin{tabular}{|p{3cm}|p{5cm}|p{4cm}|}
\hline
\multicolumn{3}{|c|}{\textbf{MoSCoW requirements for The Mobile Client}} \\
\hline
\textbf{Requirement No.} & \textbf{Requirement} & \textbf{Priority}\\
\hline
1 & Must be able to communicate with the Server Application & Must Have \\
\hline
2 & Must be able to upload files into the database through the server & Must Have \\
\hline
3 & Must be able to delete files & Must Have \\
\hline
4 & All changes must be reflected in the files stored in the server & Must Have \\
\hline
5 & The mobile UI must be simple and easy to navigate through & Should Have \\
\hline
\end{tabular}

\subsection{\underline{Server Application}}

\begin{tabular}{|p{3cm}|p{5cm}|p{4cm}|}
\hline
\multicolumn{3}{|c|}{\textbf{MoSCoW requirements for Server Application}} \\
\hline
\textbf{Requirement No.} & \textbf{Requirement} & \textbf{Priority}\\
\hline
1 & Has to be connected to a database in order to store and retrieve files & Must Have \\
\hline
2 & Must be able to communicate with the Desktop and Mobile Clients simultaneously & Must Have \\
\hline
3 & Use HTTP requests to send data to the clients & Must Have \\
\hline
4 & Handle conflicts using Rsync & Should Have \\
\hline
5 & Use security encryption to protect the data & Could Have\\
\hline
\end{tabular}

\section{\underline{Implementation}}
\subsection{\underline{Server Application}}
The server was developed using Node.js and the database used is a cloud service by MongoDB.

\subsection{\underline{Desktop Client}}
The desktop client was developed using the Electron framework; thus, the implementation primarily involved HTML, Javascript and CSS coding. To start with the foundation, Electron provides JS and HTML code (Electron, n.d.) to form a basic desktop app and webpage, respectively; therefore, we used this pre-existing code as the starting point for the development of the desktop client. Afterwards, we modified the webpage by adding a title (i.e. "Project Prime Desktop"), a file upload bar and a display of the list of server-contained files using HTML and CSS. To upload a file into the database, the desktop client would transmit a POST request to the server after the user selects a file; as a result, the file gets stored in the database. Next, in order to display the database-contained files, the client sends a GET request to the server, which relays the files to the client. In addition, a delete button is underneath each file on the desktop client, which after being pressed, deletes the corresponding file in the server; thus, removing the file from the client UI. The rationale behind this layout is to grant simplicity to the desktop application, as well as efficiency in terms of usage. 

\subsection{\underline{Mobile Client}}
The mobile client was developed using Android Studio; thus, the implementation primarily involved Java and XML.

\section{\underline{Teamwork}}
Project Prime consists of six members: Yusaf, Sandipan, Saloni, Shefali, Cameron and Manny. To balance the workload, the team was divided into three subgroups of two teammates and each subgroup was assigned to the development of one component. As a result, Yusaf and Sandipan were assigned to the desktop client, Saloni and Shefali were assigned to the server, and Manny and Cameron were assigned to the mobile client. Within these subgroups, the work was distributed between both teammates through discussion. Despite the team division, members from other subgroups were permitted to intervene in the development of a component that they weren't assigned to, in order to fix issues that the assigned subgroup couldn't correct, for example. Moving onto communication, the team remotely communicated using an instant messaging app (i.e. Whatsapp), which allowed arrangements of group sessions at a certain date-and-time, notifying teammates of open pull requests, etc. Furthermore, the sessions were conducted in booked study rooms at least once per week and these sessions involved group discussion and coding of all components. Next, the project involved Github, where the project implementation was contained in a public repository called "Project Prime Dev". In addition to Git, the team followed the feature branch workflow, where a component feature was developed inside a branch seperate to the master branch and subsequently, the former would be merged into the latter branch.

\section{\underline{Evaluation}}
To start with what went well, one positive aspect was the firm strength of team communication, as all members proactively shared their thoughts and concerns in physical meetings and in the WhatsApp group. In addition, communication was carried-out in a respectful manner at all times; thus, there were no verbal conflicts during the project. Another positive aspect was the fairness in workload distribution by creating subgroups that were assigned to the development of a specific component. The rationale behind the formation of subgroups was to avoid the possibility of one member feeling overwhelmed from having lots of work. Furthermore, by assembling subgroups, this influenced the members of the subgroup to cooperate together to build the component; thus, being in a subgroup generated a sense of teamwork. Finally, a notable positive aspect was our ability to adapt to changing circumstances. For instance, it was initially planned to store files in a SQL database. Unfortunately, we discovered SQL databases were unideal for file storage since it is limited by file size. In response, we decided to migrate to MongoDB, which is a cloud service that allows the creation of document-oriented database systems that can contain files of any size. 

\noindent Moving onto what didn't go well, our initial plan was weak as we didn't know how to approach the task, due to having no experience in developing file synchronisers. Although we met various project objectives, our commitment to the plan was low, as we rarely compared our progress to the initial plan during our weekly meetings. Another negative was the poor interactive feedback of certain features in particular components. For example, after pressing the \textit{Browse} button in the desktop application, there is no feedback (e.g. temporary colour change, mouse cursor change) to indicate the button-pressed.

\noindent Relative to the initial plan, there were differences between the rough timetable and actual progress. For instance, instead of using an SQL database, we decided to create and use a MongoDB database since it's document-oriented and has the capability to store files of any size, whereas SQL storage is limited by data type and size, and it cannot process text files properly; thus, with MongoDB being more suited to our needs, we abandoned the plan to use an SQL database and shifted to MongoDB instead. Another difference is we didn't sketch each component because we wanted to concentrate our efforts on the actual implementation of the clients and server, as well as to save time.

\noindent On to how the team worked together, the team was divided into three subgroups as initially planned, where each subgroup developed one component of the file synchronising system; as a result, the workload was divided fairly amongst all members. Throughout the majority of the project's lifetime, the team attended group meetings once per week where group discussion and coding was conducted. Along with group meetings, communication was also conducted in a whatsapp group where we scheduled groups meetings in booked study rooms at a certain date-and-time, notified each other about opened pull requests, reminders about incomplete work, etc. The reason for choosing whatsapp as our main form of remote communication was due to everyone's familiarity with the app and its simplicity (in terms of usability); thus, there was no need to use Skype since the team was comfortable with using whatsapp, in contrary to the original plan. Despite developing the system outside the meetings, one major weakness was scheduling meetings once per week since it delayed the completion of work, which could have been completed faster if we met more than once per week by completing tasks together. Overall, the team worked well together, as there were no conflicts and each member contributed their unique  skills to the group. For instance, since Yusaf possessed leadership experience, he was elected as team leader to command the group, ensure each member's involvement in the project's activities, etc.

\noindent In conclusion, this project caused the realisation that despite our computer science backgrounds, our technical prowess is still very basic and there is much for us to learn. In response to this discovery, we will commit to thoroughly relearning and practising how to use different technologies in our spare time, including programming languages and Git (e.g. BitBucket). In retrospect, if this project was repeated, our different approach would be to partner stronger members with novice ones, in terms of technological skill; thus, providing novice members with the opportunity to enhance their skills by learning whilst working on the job.   

\section{\underline{Peer Assessment}}
\begin{tabular}{|p{2cm}|p{2cm}|p{2cm}|p{2cm}|p{2cm}|p{2cm}|}
\hline
\multicolumn{6}{|c|}{\textbf{Peer Assessment of Project Prime}} \\
\hline
\textbf{Yusaf} & \textbf{Sandipan} & \textbf{Saloni} & \textbf{Shefali} & \textbf{Cameron} & \textbf{Manny} \\
\hline
17.5 & 16.5 & 16.5 & 16.5 & 16.5 & 16.5 \\
\hline
\end{tabular}
	
\section{\underline{References}}
\begin{itemize}
\item Madsen, S. (2017) \textit{How to Prioritize with the MoSCoW Technique} [online] Available at: https://www.projectmanager.com/training/prioritize-moscow-technique [Accessed on 10 March 2019]
\item Electron (n.d.) \textit{Writing your First Electron App | Electron} [online] Available at: https://electronjs.org/docs/tutorial/first-app [Accessed on 12 March 2019]
\end{itemize}
\end{document}